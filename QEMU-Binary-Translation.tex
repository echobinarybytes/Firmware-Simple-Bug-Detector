\documentclass[12pt]{article}

\usepackage[UTF8]{ctex}

\title{QEMU \- Binary Translation}
\begin{document}

\maketitle

\section{introduction of QEMU}

\subsection{Simulation vs Emulation}
what is the difference between \textbf{Simulation}  and \textbf{Emulation}, These are both cover the act of mimicking\footnote{mimicking 模仿} a real thing in a virtual environment.
\begin{itemize}
    \item \textbf{Simulation} is for analysis and study, \textbf{Emulation} is for usage as substitue\footnote{substitue: 替代物}. It means we use \textbf{Emulation} to do things for real life, while \textbf{Simulation} just be virtual, and can not do real life jod for us. For example, 
    \item the \textbf{Emulation} model can run realistic tests of the control system, by accurately represent the real automation system in great detail, and it rarely run faster than real time. On the other hand, \textbf{Simulation} models don't to be to much detailed, and they are useful if they can run many times faster than real time.
\end{itemize}

\subsection{usage of QEMU}
\textbf{QEMU} is popular used for \textit{cross-compilation development environments}, \textit{Virtulization, device emulation, for kvm, Android Emulator} 

\subsection{Dynamic Translation}







\end{document}
